

\documentclass{article}
\usepackage[utf8]{inputenc}
\usepackage{setspace}
\usepackage{ mathrsfs }
\usepackage{amssymb} %maths
\usepackage{amsmath} %maths
\usepackage[margin=0.2in]{geometry}
\usepackage{graphicx}
\usepackage{ulem}
\setlength{\parindent}{0pt}
\setlength{\parskip}{10pt}
\usepackage{hyperref}
\usepackage[autostyle]{csquotes}

\usepackage{cancel}
\renewcommand{\i}{\textit}
\renewcommand{\b}{\textbf}
\newcommand{\q}{\enquote}




\begin{document}

\begin{huge}

{\setstretch{0.0}{
Scope [ Javascript Version ]

I wrote the first version of Scope in C, basically for speed (lots of layers). I wanted to see the evolving model graphed like the typical function $f : \Bbb R \to \Bbb R$.
Eventually I added an illustration of evolving network itself, graphing magnitude as thickness and sign as color. Every detail (such as the activations of individual neurons) could be modified \q{live.}

All this detail is too messy for the Browser, so I wrote this browser version to offer the basic experience. While I have used dual numbers in some programs to train models, this version of Scope uses classic back-propagation. Some versions featured 3 functions learning on the same graph, and it's not hard to add them back in if one is interested. 
 }}
\end{huge}
\end{document}
